%%
%% 研究報告用スイッチ
%% [techrep]
%%
%% 欧文表記無しのスイッチ(etitle,jkeyword,eabstract,ekeywordは任意)
%% [noauthor]
%%

\documentclass[submit,techrep]{ipsj}
%\documentclass[submit,techrep,noauthor]{ipsj}



\usepackage[dvips]{graphicx}
\usepackage{latexsym}

\def\Underline{\setbox0\hbox\bgroup\let\\\endUnderline}
\def\endUnderline{\vphantom{y}\egroup\smash{\underline{\box0}}\\}
\def\|{\verb|}

\setcounter{巻数}{53}%vol53=2012
\setcounter{号数}{10}
\setcounter{page}{1}


\begin{document}


\title{無意識参加型センシングシステムの有効性の検証}

\etitle{Verification of the effectiveness of an Unconscious Participatory Sensing System}

%\affiliate{IPSJ}{情報処理学会\\
%IPSJ, Chiyoda, Tokyo 101--0062, Japan}


%\paffiliate{JU}{情報処理大学\\
%Johoshori Uniersity}

\affiliate{GSBACSA}{愛知工業大学大学院 経営情報科学研究科\\
Graduate School of Business Administration and Computer Science, Aichi Institute of Technology}
\affiliate{FISA}{愛知工業大学 情報科学部\\Faculty of Information Science, Aichi Institute of Technolog}
\affiliate{NTTDOCOMO}{NTTドコモ先進技術研究所\\Research Laboratories, NTT DOCOMO, Inc.}

\author{水上 貴晶}{Takamasa Mizukami}{GSBACSA}[b14723bb@aitech.ac.jp]
\author{堀 将之}{Masayuki Hori}{FISA}%[hori@pluslab.org]
\author{古謝 佑次}{Yuji Koja}{FISA}%[hori@pluslab.org]
%\author{大村 和徳}{Kazunori Omura}{FISA}[oremura@gmail.com]
%\author{河合 亮太}{Ryota Kawai}{FISA}
%\author{河南 光}{Hikaru Kawanami}{FISA}
\author{土井 千章}{Chiaki Doi}{NTTDOCOMO}%[chiaki.doi.tf@nttdocomo.com]
\author{太田 賢}{Ken Ohta}{NTTDOCOMO}%[ootaken@nttdocomo.com]
\author{稲村 浩}{Hiroshi Inamura}{NTTDOCOMO}%[inamura@nttdocomo.com]
\author{梶 克彦}{Katsuhiko Kaji}{FISA}%[kaji@aitech.ac.jp]
\author{内藤 克浩}{Katsuhiro Naito}{FISA}%[naito@pluslab.org]
\author{菱田 隆彰}{Takaaki Hishida}{FISA}%[hishida@aitech.ac.jp]
\author{水野 忠則}{Tadanori Mizuno}{FISA}%[mizuno@mizulab.net]


\begin{abstract}
スマートフォンの普及を背景に,一般ユーザの持つスマートフォンを利用して高密度,広範囲の環境情報を収集する参加型センシングの実現が期待をされている.しかし,参加型センシングは多くの参加者が存在する事を前提としており,実用的なシステムを構築するためには手軽に参加してもらい,より多くの参加者が必要である.
著者らは参加者にセンシング意識させない無意識参加型センシングの手法とプロトタイプ実装を行ってきた.
本稿では,従来の参加型センシングと無意識参加型センシングのシミュレータを構築し,センサデバイスが定点観測しているセンサデータの報告数を定量的に評価した.愛知工業大学をモデルとした参加者を自由に動かしたシナリオにおいてセンサデバイスと参加者の持つスマートフォンとで無線通信を行いデータ取得を実現した.シミュレーションの結果,無意識参加型センシングの特性と検証結果を報告する.
\end{abstract}

%ストーリー
%無意識の必要性を記述(参加型センシングと比較)
%無意識参加型センシングの概要
	%想定環境
%事前実験
%シミュレーション評価
%まとめ

%章立て
%はじめに
%関連研究
%無意識参加型センシングの概要
	%想定環境
%事前実験
%シミュレーション評価
%まとめ


%
%\begin{jkeyword}
%情報処理学会論文誌ジャーナル,\LaTeX,スタイルファイル,べからず集
%\end{jkeyword}
%
%\begin{eabstract}
%This document is a guide to prepare a draft for submitting to IPSJ
%Journal, and the final camera-ready manuscript of a paper to appear in
%IPSJ Journal, using {\LaTeX} and special style files.  Since this
%document itself is produced with the style files, it will help you to
%refer its source file which is distributed with the style files.
%\end{eabstract}
%
%\begin{ekeyword}
%IPSJ Journal, \LaTeX, style files, ``Dos and Dont's'' list
%\end{ekeyword}

\maketitle

%1
\section{はじめに}

%研究背景を記述
%参考文献を引用しながら述べる

近年,安価で高精度な物理センサの普及によりセンサネットワークにおける研究が盛んに行われている.センサネットワークは様々な情報を収集するネットワークシステムとして着目されている\cite{Viani}.物理センサにより温度や湿度,照度などの環境情報を定量的に取得する事が可能である.また,スマートフォンの普及によって一般者が携帯するスマートフォンを用いたセンシングに着目されている.スマートフォンには加速度センサやGPSセンサなど様々な物理センサが搭載されている.そこでセンシングデバイスとしてスマートフォンを活用した参加型センシングに関する研究が行われている\cite{Nicholas}.

対象とする参加型センシングでは,センシングの依頼者が街中に散らばるスマートフォンを持つ参加者(一般者)に対して,センシングしたい各地点にセンシングを依頼するものと想定している\cite{Burke}.従って,前提条件としてセンシングをしたい各地点にセンシング参加者が多数存在している事である.センシング処理に一般者が関与することから,測定場所,測定対象,測定方法などの自由度が大きく,一般者の持つスマートフォンを用いた新たな情報収集手法として注目されている.


参加型センシングで収集する情報は大別して,人による評価情報を収集する場合\cite{Lam}と物理センサなどを用いて定量的に測定可能な事象を収集する場合\cite{Shinohara}がある.人による評価情報とは物理センサでは測定できない,人間の持つ五感を活かした主観的かつ定性的な情報を収集し従来のセンシングでは取得できなかった情報を広範囲,高密度なセンシング期待されている.一方,物理センサを用いた定量的に測定可能な事象を対象とした場合,参加者のスマートフォンに搭載されているセンサを利用する場合する事となる.そのため,システムから依頼を受けた参加者は,測定対象場所に移動しスマートフォンに搭載されているセンサ情報を取得する事になる.
しかし,現実的なセンシングを考えた場合,スマートフォンに搭載されている,温度,気圧などのセンサは,スマートフォンの機種が異なる場合やモデルによって,実装されているセンサが異なり,センサ情報の精度及び実装環境が異なる事が予測される.このような条件では,同一環境において測定を行ったとしても,スマートフォンの機種やモデルにより異なる測定値が得られる事が考えられる.そのため,スマートフォン内臓センサで利用可能なものは,加速度やGPSなどの比較的,機種やモデルの違いに影響を受けない測定値に限定されると予測される.結果として,特定の環境を継続的に測定する場合には,スマートフォンを通信回線として利用する一方で,測定自体は別デバイスを用いて測定値の精度を高める必要がある\cite{Li}.本稿では物理センサを用いた定量的に測定可能な事象を対象とした参加型センシングについて取り上げる.


参加型センシングにおいては参加者が自主的にセンシング処理に関与する必要があるため,多くの研究において参加者が積極的にセンシング処理に関与するための動機付け手法が検討されている\cite{Yoshitaka}.


また,参加型センシングとソーシャルメディアの情報を複合的に処理を行う手法なども提案されている\cite{Demirbas}.前述した通り,様々な研究が続けられているが,現状の参加型センシングでは,一般者を想定したシステムとまでは確立できておらず,センシングに興味を持つ一部の一般者を想定としている状況である.

本稿では,著者らが先行研究\cite{Mizukami}で行っている無意識参加型センシングをシミュレータScenargie\cite{Scenargie}を用いて,従来の参加型センシングとこれまでに提案してきた無意識参加型センシングのモデルを評価した.シミュレーションでは都市部を想定したシナリオを用いて,参加型センシングによるセンサデータ収集量と無意識参加型センシングによるデータ報告数を比較した結果,無意識参加型センシングにおいて,人通りの多い場所付近でのセンシングポイントにてセンシングデータをより多く収集できる事を確認できた.



%情報処理学会では,基幹論文誌として論文誌ジャーナルの発行を行っている.こ
%れまで論文誌ジャーナル編集委員会では,論文誌ジャーナルの論文掲載時のフォー
%マットとしてA4横型2段組を採用してきたが,会員からの多くの要望に基づき,
%A4縦型2段組に変更することにした.また,これまでは投稿時と掲載時の形式が
%異なっていたが,今回のフォーマット変更に合わせて,投稿時も掲載時と同様の
%A4縦型2段組で受け付けることにした.


%これに伴い,\LaTeX のスタイルファイルも新しいものに変更した.本稿では,
%まずそのスタイルファイルを用いた論文のフォーマットに関して述べる.新たな
%スタイルファイルでは,極力特別なコマンドは使わずに,標準的な \LaTeX のス
%タイルを踏襲している.論文フォーマットに関しては,\ref{sec:format}~章で
%後述する指針に従って頂くが,そこに規定されていること以外は標準的な\LaTeX
%のコマンドをそのまま使うことができる.本稿は,そのスタイルファイルを実際
%に使っているので,論文執筆の際に参考にされたい.


%\footnotetext{本文は実際には論文誌ジャーナル編集委員会で作成したものである.}

%また,論文誌ジャーナル編集委員会では,論文の執筆する際に,著者がするべき
%こと,するべきでないことを「べからず集」としてまとめた.本稿の後半に,論
%文の内容に関する指針になるように,「べからず集」の内容をチェックリストと
%してつけているので,投稿する前の内容のチェックに利用されたい.

%2
\section{従来の一般的なセンシング方式と提案方式}
%従来の参加型センシングのモデルと比較して
%提案手法の無意識参加型センシングの概要を記述
本研究では,通信インフラを持たず,音や照度,温度,湿度といった物理的現象を定期的にセンシングする固定されたセンサデバイスが環境情報を収集したい場所に複数設置されている事を想定する.またセンシング参加者も多数存在している事を想定する.センシング参加者が携帯するスマートフォンを利用して定点観測しているセンサデバイスに近づき,無線通信を用いてセンサデータを収集し,スマートフォンの通信回線を用いて報告を行うモデルについて考える.

%2.1
\subsection{参加型センシング}
%参加型センシングの一般的なモデルの図を描く
本稿が扱う参加型センシングでは,クライアントとスマートフォンを携帯している複数の参加者が存在している.クライアントは参加者に対してインターネットを介して,特定のセンシングの依頼を行う.依頼の例として以下に述べる.「ある地域内の天気を報告してください」「現在いる場所の温度を報告してください」といったものが例としてある.さらにより正確な環境情報を収集するため本稿ではセンシングするポイントには定点観測しているセンサデバイスが存在している.したがって参加者はクライアントの依頼に対してセンシングポイントに移動し,確率的にセンシングへの参加・不参加をするもととする.図\ref{Participatory}に想定する参加型センシングの概要図を示す.

\begin{figure}[t]
 \begin{center}
  \includegraphics[keepaspectratio, width=80mm,height=40mm]{Participatory_sensing.eps}
 \end{center}
 \caption{Participatory sensing system summary}
 \label{Participatory}
\end{figure}


%2.3
\subsection{提案方式}
%無意識参加型センシングの一般的なモデルの図を描く
先行研究で提案およびプロトタイプ実装している無意識参加型センシングは,定点観測しているセンサデバイスに近距離無線通信機能(Bluetooth Low Energy)とiBeacon\cite{iBeacon}の技術を組み合わせたセンサデバイスを利用する.参加型センシングと異なる点はクライアントが参加者に対して依頼をするわけではなく,センサデバイスが能動的に参加者に対してセンシング及びデータ収集を依頼する.つまり,偶然近接に存在するスマートフォンとスマートフォンが連携する事により,定点観測された情報を報告する.これにより,参加者は意識的にセンシングポイントに移動を行ったり,スマートフォンを操作して意識的にセンシングに参加を行う事なく無意識的にセンシングに参加が可能である.図\ref{UnconsciousP}に提案手法である無意識参加型センシングの概要図を示す.

\begin{figure}[t]
 \begin{center}
  \includegraphics[keepaspectratio, width=80mm,height=40mm]{UnconsciousParticipatorySensing.eps}
 \end{center}
 \caption{Unconscious participatory sensing system summary}
 \label{UnconsciousP}
\end{figure}


%本稿に従って用意した投稿用原稿の \LaTeX ソースからpdfファイルを作成し,
%Adobeのpdf readerで読めることを確認した後,

%2.3
%\subsection{最終原稿の作成とファイルの送付}

%投稿した論文の採録が決定したら,査読者からのコメントなどにしたがって原稿
%を修正し,著者紹介など投稿時になかった項目があれば追加する.また図表など
%のレイアウトも最終的なものとする.なお後の校正の手間を最小にするために,
%この段階で記述の誤りなどを完全に除去するように綿密にチェックして頂きたい.

%2.4
%\subsection{著者校正・組版・出版}

%学会では用語や用字を一定の基準に従って修正することがある.また \LaTeX の
%実行環境の差異などによって著者が作成したハードコピーと実際の組版結果が微
%妙に異なることがある.これらの修正や差異が問題ないかを最終的に確認するた
%めに,著者にゲラ刷りが送られるので,もし問題があれば朱書によって指摘して
%返送する.なお{\bf この段階での記述誤りの修正は原則として認められない}の
%で,原稿送付時に細心の注意を払っていただきたい.

%3
\section{無意識参加型センシングの詳細}

%3.1

\subsection{無意識参加型センシングの利点}
従来の一般的な参加型センシングでは参加者はクライアントからの依頼を受けないため,クライアントは参加者の位置を事前に把握し通知行わなくてよいため,参加者の位置情報などプライバシ情報が不要である.従って,センシングを依頼するためのトラフィックコストの削減が可能と考える.iBeaconの機能を用いる事によってアプリケーションをバックグラウンドで自動的に立ち上げる事ができるため参加者はスマートフォンを操作したり,特定のセンシング場所に赴く事を必要としないでセンシングに参加できる事から,参加者の負担も減少させる事ができると考える.一方センサ側の利点としては同一センサを利用した定点観測が可能であり,特定の場所の継続的な測定が可能である.また,センシングデータの報告にはスマートフォンの回線を利用して報告するため,センサデバイスの設置の際,ネットワークの有無に関わらず,容易に設置が可能である.上記の通り,センシング及びセンシングデータ収集を依頼する視点を変更する事によって,参加者やセンサデバイスに利点と特徴が生まれる.

%\label{config}

%ファイルは次のようになる.下線部は投稿時に省略可能なもの.またトランザク
%ション特有コマンドについては \ref{sig}~節を参照されたい.

%4.1
%\subsection{オプション・スタイル}

%\label{option} \|\documentclass{ipsj}|のオプション\footnote{研究会用のオ

%4.2
%\subsection{表題・著者名等}

%表題,著者名とその所属,および概要を前述のコマンドや環境により{\bf 和文と
%英文の双方について}定義した後,\|\maketitle| によって出力する.

%4.2.1
%\subsubsection{表題}

%表題は,\|\title| および \|\etitle| で定義した表題はセンタリングされる.
%文字数の多いものについては,適宜 \|\\| を挿入して改行する.

%4.2.2
%\subsubsection{著者名・所属}

%4.2.3
%\subsubsection{概要}

%和文の概要は \|abstract| 環境の中に,

%4.2.4
%\subsubsection{キーワード}

%英文の概要は \|ekeyword| 環境の中に,それぞれ1\UTF{FF5E}5語記述する.

%4.3
%\subsection{本文}

%4.3.1
%\subsubsection{見出し}

%4.3.2
%\subsubsection{行送り}


%4.3.3
%\subsubsection{フォントサイズ}

%4.3.4
%\subsubsection{句読点}

%句点には全角の「.」,読点には全角の「,」を用いる.ただし英文中や数式中
%で「.」や「,」を使う場合には,半角文字を使う.「.」や「,」は使わない.

%4.3.5
%\subsubsection{全角文字と半角文字}

%全角文字と半角文字の両方にある文字は次のように使い分ける.

%\begin{enumerate}
%\item 括弧は全角の「(」と「)」を用いる.但し,英文の概要,図表見出し,
%書誌データでは半角の「(」と「)」を用いる.

%4.3.6
%\subsubsection{箇条書}

%箇条書に関する形式を特に定めていない.場合に応じて標準的な \|enumerate|,
%\|itemize|, \|description| の環境を用いてよい.

%4.3.7
%\subsubsection{脚注}

%脚注は \|\footnote| コマンドを使って書くと,ページ単位に\footnote{脚注の
%例.}や\footnote{二つめの脚注.}のような参照記号とともに脚注が生成される.

%4.3.8
%\subsubsection{OverfullとUnderfull}

%4.4
%\subsection{数式}\label{sec:Item}

%4.4.1
%\subsubsection{本文中の数式}

%本文中の数式は \|$| と \|$|, \|\(| と \|\)|, あるいは \|math| 環境のいず
%れで囲んでもよい.

%4.4.2
%\subsubsection{別組の数式}

%別組数式(displayed math)については \|$$| と \|$$| は使用せずに,\|\[| と
%\|\]| で囲むか,\|displaymath|, \|equation|, \|eqnarray| のいずれかの環
%境を用いる.これらは
%

%4.4.3
%\subsubsection{eqnarray環境}


%4.4.4
%\subsubsection{数式のフォント}

%4.5
%\subsection{図}

%\begin{figure*}[tb]
%\setbox0\vbox{\large
%\hbox{\|\begin{figure*}[t]|}
%\hbox{\quad \|<|図本体の指定\|>|}
%\hbox{\|\caption{<|和文見出し\|>}|}
%\hbox{\|\ecaption{<|英文見出し\|>}|}
%\hbox{\|\label{| $\ldots$ \|}|}
%\hbox{\|\end{figure*}|}}
%\centerline{\fbox{\hbox to.9\textwidth{\hss\box0\hss}}}
%\caption{2段幅の図}
%\ecaption{Double column figure.}
%\label{fig:double}
%\end{figure*}

%4.6
%\subsection{表}

%また,表の上に和文と英文の双方の見出しを, \|\caption|と \|\ecaption| で
%指定する.表の参照は \|\tabref{<|ラベル\|>}| を用いて行なう.

%\begin{table}[tb]
%\caption{表の例}
%\ecaption{An Example of Table.}
%\label{tab:example}
%\hbox to\hsize{\hfil
%\begin{tabular}{l|lll}\hline\hline
%& column1 & column2 & column3 \\\hline
%row1 &	item 1,1 & item 2,1 & ---\\
%row2 &	---      & item 2,2 & item 3,2 \\
%row3 &	item 1,3 & item 2,3 & item 3,3 \\
%row4 &	item 1,4 & item 2,4 & item 3,4 \\\hline
%\end{tabular}\hfil}
%\end{table}




%4.7
%\subsection{参考文献・謝辞}

%4.7.1
%\subsubsection{参考文献の参照}

%本文中で参考文献を参照する場合には%,%参考文献番号が文中の単語として使われ
%る場合と,そうでない参照とでは,使用する文字の大きさが異なる.前者は
%\|\Cite|により参照し,後者は
%\|\cite|により参照する.たとえば;
%\|\cite|を使用する.参照されたラベルは自動的にソートされ,
%\|[]|でそれぞれ区切られる.
%
%\begin{quote}
%文献 \|\cite{companion,okumura}| は \LaTeX の総合的な解説書である.
%\end{quote}
%
%と書くと;
%
%\begin{quote}
%文献\cite{companion,okumura}は \LaTeX の総合的な解説書である.
%\end{quote}
%
%が得られる.

%4.7.2
%\subsubsection{参考文献リスト}


%4.8
%\subsection{著者紹介}

%6
%\section{まとめ}

%論文の内容について,論文誌ジャーナル編集委員会で作成した「べからず集」を
%以下に示す.投稿前のチェックリストとして利用頂きたい.これ以外にも,査読
%者用,メタ査読者用の「べからず集」\cite{webpage2}も公開しているので,参
%照されたい.また,作文技術に関する \cite{book1, book2, book3, book4}のよ
%うな書籍も参考になる.

%5.1
%\subsection{書き方の基本}

%5.2
%\subsection{新規性と有効性を明確に示す}

%\begin{itemize}
% \item[$\Box$] 在来研究との関連,研究の動機,ねらい等が明確に説明されて
%	       いないのは再考を要する.
% \item[$\Box$] 既知/公知の技術が何であって,何を新しいアイデアとして提
%	       案しているのかが書かれていないのは再考を要する.
%\item[$\Box$] 十分な参考文献は新規性の主張に欠かせない.
% \item[$\Box$] 提案内容の説明が,概念的または抽象的な水準に終始していて,
%	       読者が提案内容を理解できない(それだけで新規性が感じられ
%	       ないもの)のは再考を要する.
% \item[$\Box$] 論文で提案した方法の有効性の主張がない,またはきわめて貧
%	       弱なのは再考を要する.
%\end{itemize}

%5.3
%\subsection{書き方に関する具体的な注意}

%5.4
%\subsection{参考文献}

%5.5
%\subsection{二重投稿}


%5.6
%\subsection{他の人に読んでもらう}


%5.7
%\subsection{その他}

%6
%\section{おわりに}

\begin{thebibliography}{10}

%\bibitem{latex}
%Lamport, L.: {\em A Document Preparation System \LaTeX User's Guide \&
%  Reference Manual}, Addison Wesley, Reading, Massachusetts (1986).
% (Cooke, E., et al.訳:文書処理システム \LaTeX,アスキー出版局
%  (1990)).

%\bibitem{total}
%伊藤和人: \LaTeX トータルガイド,秀和システムトレーディング (1991).
%\bibitem{nodera}
%野寺隆志:楽々 \LaTeX,共立出版 (1990).

\if0
\bibitem{okumura}
奥村晴彦:改訂第5版 \LaTeXe 美文書作成入門,
技術評論社(2010).

\bibitem{companion}
Goossens, M., Mittelbach, F. and Samarin, A.:
{\it The LaTeX Companion},
Addison Wesley, Reading, Massachusetts (1993).

\bibitem{book1}
木下是雄:
理科系の作文技術,
中公新書(1981).

\bibitem{book2}
Strunk W. J. and White E.B.:
{\it The Elements of Style, Forth Edition},
Longman (2000).

\bibitem{book3}
Blake G. and Bly R.W.:
{\it The Elements of Technical Writing},
Longman (1993).

\bibitem{book4}
Higham N.J.:
{\it Handbook of Writing for the Mathematical Sciences},
SIAM (1998).

\bibitem{webpage1}
情報処理学会論文誌ジャーナル編集委員会:
投稿者マニュアル(online),
\urlj{http://www.ipsj.or.jp/journal /submit/manual/j\_manual.html}
(2007.04.05).

\bibitem{webpage2}
情報処理学会論文誌ジャーナル編集委員会:
べからず集(online),
\urlj{http://www.ipsj.or.jp/journal/manual /bekarazu.html}
(2011.09.15).
\fi

%1
\bibitem{Viani}
F. Viani, P. Rocca, G. Oliveri, and A. Massa:
Pervasive remote sensing through WSNs, In Antennas and Propagation (EUCAP), 2012 6th European Conference on, pp. 49-50. IEEE, 2012.

%2
%\bibitem{Lane}
%N. D. Lane, S. B. Eisenman, M. Musolesi, E. Miluzzo, and A. T. Campbell:
%Urban sensing systems: opportunistic or participatory?,
%In Proceedings of the 9th Workshop on Mobile Computing Systems and Applications, HotMobile'08, pp. 11-16, 2008.

%3
\bibitem{Nicholas}
N. D. Lane, E. Miluzzo, H. Lu, D. Peebles, T. Choudhury, and A. T. Campbell:
A survey of mobile phone sensing,
IEEE Communications Magazine, Vol. 48, No. 9,
September 2010.

%4
%\bibitem{Mohan}
%P. Mohan, V. N. Padmanabhan, and R. Ramjee:
%Nericell: rich monitoring of road and traffic conditions using mobile smartphones,
%SenSys'08: Proceedings of the 6th ACM Conference on Embedded Network Sensor Systems,
%November 2008.

%5
\bibitem{Burke}
J. Burke, D. Estrin, M. Hansen, A. Parker, N. Ramanathan, S. Reddy, and M. B. Srivastava:
Participatory sensing, Mobile Device Centric Sensor Networks and Applications,
In Workshop on World-Sensor-Web (WSW),
pp. 117-134, 2006.

%6
%\bibitem{Wang}
%D. Wang, M. T. Amin, S. Li, T. Abdelzaher, L. Kaplan, S. Gu, C. Pan, H. Liu, C. C. Aggarwal, R. Ganti, X. Wang, P. Mohapatra, B. Szymanski, and H. Le:
%Using Humans as Sensors: An Estimation-theoretic Perspective,
%In
%IPSN'14 Proceedings of the 13th International Symposium on Information Processing in Sensor Networks,
%pp. 35-46, 2014.

%7
\bibitem{Niforatos}
%E. Niforatos, A. Vourvopoulos, M. Langheinrich, P. Campos, and A. Doria:
%Atmos: a hybrid crowdsourcing approach to weather estimation,
%UbiComp'14: Proceedings of the 2014 ACM International Joint Conference on Pervasive and Ubiquitous Computing,
%September 2014.

%8
\bibitem{Lam}
A. H. Lam, Y. Yuan, and D. Wang:
An occupant-participatory approach for thermal comfort enhancement and energy conservation in buildings, The 5th International Conference on Future Energy Systems (e-Energy'14), June 2014.

%9
%\bibitem{Budde}
%M. Budde, R. E. Masri, T. Riedel, and M. Beigl:
%Enabling Low-Cost Particulate Matter Measurement for Participatory Sensing Scenarios,
%In Proceedings of the 12th International Conference on Mobile and Ubiquitous Multimedia (MUM'13), December 2013.

%10
%\bibitem{Hull}
%B. Hull, V. Bychkovsky, Y. Zhang, K. Chen, M. Goraczko, A. Miu, E. Shih, H. Balakrishnan, and S. Madden:
%CarTel: a distributed mobile sensor computing system,
%In SenSys'06, pp. 125-138, 2006.

%11
%\bibitem{Zeiger}
%F. Zeiger and M. Huber:
%Demonstration abstract: participatory sensing enabled environmental monitoring in smart cities, The 13th International Symposium on Information Processing in Sensor Networks  (IPSN'14), April 2014.

%12
\bibitem{Shinohara}
篠原 雅貴,田島 誠也,中下 岬,近藤 亮磨,岩井 将行,"気体情報の時系列解析による集合施設の生活環境・活動状況推測システム",マルチメディア,分散協調とモバイルシンポジウム2014論文集,Vol.2014,pp.1892-1897,2014.
%巻,号,pp

%13
\bibitem{Li}
L. Li, Y. Zheng, and L. Zhang:
Demonstration abstract: PiMi air box: a cost-effective sensor for participatory indoor quality monitoring,
IPSN'14 Proceedings of the 13th International Symposium on Information Processing in Sensor Networks,
pp. 327-328, 2014.

%14
%\bibitem{Tomasic}
%A. Tomasic, J. Zimmerman, A. Steinfeld, and Y. Huang:
%Motivating Contribution in a Participatory Sensing System via Quid-Pro-Quo,
%In CSCW'14 Proceedings of the 17th ACM Conference on Computer Supported Cooperative Work \& Social Computing,
%pp. 979-988, 2014.

%15
%\bibitem{Zhang}
%D. Zhang, H. Xiong, L. Wang, and G. Chen:
%CrowdRecruiter: selecting participants for piggyback crowdsensing under probabilistic coverage constraint,
%UbiComp'14: Proceedings of the 2014 ACM International Joint Conference on Pervasive and Ubiquitous Computing,
%September 2014.

%16
\bibitem{Yoshitaka}
上山 芳隆 ,玉井 森彦 ,安本 慶一 ,"ユーザ参加型センシングにおけるゲーミフィケーションに基づくインセンティブ機構の提案",研究報告モバイルコンピューティングとユビキタス通信(MBL),Vol.66,No.12,pp.1-6,2013.
%巻,号,pp

%17
\bibitem{Demirbas}
M. Demirbas, M. A. Bayir, C. G. Akcora, Y. S. Yilmaz, and H. Ferhatosmanoglu:
Crowd-sourced sensing and collaboration using twitter,
WOWMOM'10 Proceedings of the 2010 IEEE International Symposium on A World of Wireless, Mobile and Multimedia Networks (WoWMoM),
pp. 1-9, 2010.

%18
\bibitem{Mizukami}
T. Mizukami, K. Naito, C. Doi, T. Nakagawa, K. Ohta, H. Inamura, T. Hishida, and T. Mizuno:
Fundamental Design for a Beacon Device Based Unconscious Participatory Sensing System,
International MultiConference of Engineers and Computer Scientists 2015, Vol. 2 2015.

%19
\bibitem{Scenargie}
Space-Time Engineering: Scenargie, [Online]. Available: http://www.spacetime-eng.com/en/products.html. Retrieved October 2015.

%20
\bibitem{iBeacon}
iBeacon for Developers,\\https://developer.apple.com/ibeacon/,
Retrieved October 2014.

%22
\bibitem{eStat}
e-Stat,\\http://www.e-stat.go.jp/SG1/estat/eStatTopPortal.do,
Retrieved October 2016.

%23
\bibitem{General}
総務省 情報通信の現況・政策の動向,\\
http://www.soumu.go.jp/johotsusintokei/\\
whitepaper/ja/h26/html/nc253120.html,
Retrieved October 2016.

%24
\bibitem{JMR}
JMR生活研究所 イノベーター理論,\\
http://www.jmrlsi.co.jp/knowledge/yougo/my02/my0219.html,
Retrieved October 2016.


\end{thebibliography}

\end{document}
